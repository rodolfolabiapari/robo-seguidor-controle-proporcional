\section{Códigos}
	\subsection{Código em Linguagem Lua}
	\subsubsection{init.lua}
		\begin{minted}
			[
			frame=lines,
			framesep=2mm,
			tabsize=3,
			breaklines=true,
			baselinestretch=1.2,
			fontsize=\scriptsize,
			linenos
			]{lua}
print("\n")
print("ESP8266 Started")

local exefile="sta"
local luaFile = {exefile..".lua","yun.lua"}
for i, f in ipairs(luaFile) do
	if file.open(f) then
		file.close()
		print("Compile File:"..f)
		node.compile(f)
		print("Remove File:"..f)
		file.remove(f)
	end
end

if file.open(exefile..".lc") then
	dofile(exefile..".lc")
else
	print(exefile..".lc not exist")
end

exefile=nil;luaFile = nil
collectgarbage()
		\end{minted}

		\subsubsection{sta.lua}
			\begin{minted}
				[
				frame=lines,
				framesep=2mm,
				tabsize=3,
				breaklines=true,
				baselinestretch=1.2,
				fontsize=\scriptsize,
				linenos
				]{lua}
print("Ready to Set up wifi mode")
wifi.setmode(wifi.STATION)
local ssid = "MERCURY_1013"
local psw = "123456789"
print("Conneting to "..ssid)
wifi.sta.config(ssid,psw)--ssid and password
wifi.sta.connect()
local cnt = 0
gpio.mode(0,gpio.OUTPUT);
tmr.alarm(3, 1000, 1, function()
	if (wifi.sta.getip() == nil) and (cnt < 20) then
		print("->")
		cnt = cnt + 1
		if cnt % 2 ==1 then
			gpio.write(0,gpio.HIGH);
		else
			gpio.write(0,gpio.LOW);
		end
	else
		tmr.stop(3)
		if (cnt < 20) then
			print("Config done, IP is "..wifi.sta.getip())
			cnt = nil;ssid=nil;psw=nil;
			collectgarbage();
			if file.open("yun.lc") then
				dofile("yun.lc")
			else
				print("yun.lc not exist")
			end
		else
			print("Wifi setup time more than 20s, Pls verify\r\nssid:"..ssid.." psw:"..psw.."\r\nThen re-download the file.")
			cnt=cnt+1;
			tmr.alarm(1, 300, 1, function()
				if cnt % 2 ==1 then
					gpio.write(0,gpio.HIGH);
				else
					gpio.write(0,gpio.LOW);
				end
			end)
		end
	end
end)
			\end{minted}

			\subsubsection{yun.lua}
				\begin{minted}
					[
					frame=lines,
					framesep=2mm,
					tabsize=3,
					breaklines=true,
					baselinestretch=1.2,
					fontsize=\scriptsize,
					linenos
					]{lua}
--yun coding demo
--Created @ 2015/05/27 by Doit Studio
--Modified: null
--http://www.doit.am/
--http://www.smartarduino.com/
--http://szdoit.taobao.com/
--bbs: bbs.doit.am

print("Start yun")

gpio.mode(0,gpio.OUTPUT);--LED Light on
gpio.write(0,gpio.LOW);

local deviceID = "doitCar"
local timeTickCnt = 0
local fileName = "yunRemote"
local conn;
local flagClientTcpConnected=false;
print("Start TCP Client");
tmr.alarm(3, 5000, 1, function()
	if flagClientTcpConnected==true then
		timeTickCnt = timeTickCnt + 1;
		if timeTickCnt>=60 then --every 300 seconds send "cmd=keep\r\n" to server
			timeTickCnt = 0;
			conn:send("cmd=keep\r\n");
		end
	elseif flagClientTcpConnected==false then
	print("Try connect Server");
	conn=net.createConnection(net.TCP, false)
	conn:connect(7500,"198.199.94.16");
		conn:on("connection",function(c)
			print("TCPClient:conneted to server");
			conn:send("cmd=subscribe&topic="..deviceID.."\r\n");
			flagClientTcpConnected = true;timeTickCnt = 0;
		end) --connection
		conn:on("disconnection",function(c)
			flagClientTcpConnected = false;
			conn=nil;collectgarbage();
		end) --disconnection
		conn:on("receive", function(conn, m)
			if string.sub(m,1,5)=="__B__" then
				file.remove(fileName..".lua")
				file.open(fileName..".lua", "w" )
				conn:send("cmd=next\r\n");--start fetching prog file
			elseif string.sub(m,1,5)=="__E__" then --finish fetching
				file.close()
				collectgarbage();
				node.compile(fileName..".lua");
				file.remove(fileName..".lua");
				dofile(fileName..".lc")
			else --the file context
				print("Recieve:"..m)
				file.writeline(m);
				conn:send("cmd=next\r\n");--continue fetching
			end
			collectgarbage();
		end)--receive
	end
end)
				\end{minted}

				\subsubsection{cloud}
					\begin{minted}
						[
						frame=lines,
						framesep=2mm,
						tabsize=3,
						breaklines=true,
						baselinestretch=1.2,
						fontsize=\scriptsize,
						linenos
						]{c}
print("start flash led")
cnt = 10
while cnt>0 do
	gpio.write(0,gpio.LOW)
	tmr.delay(1000*1000) --1second
	gpio.write(0,gpio.HIGH)
	tmr.delay(1000*1000)--1second
	cnt = cnt - 1;
end
print("yun demo finish!")
					\end{minted}


	\subsection{Código em Linguagem Arduino}
	\begin{minted}
		[
		frame=lines,
		framesep=2mm,
		tabsize=3,
		breaklines=true,
		baselinestretch=1.2,
		fontsize=\scriptsize,
		linenos
		]{C}

// Cálculos usados para calcular média
long media_sensores;
int somatorio_sensores;

// Posicao do veículo
int posicao;

int quant_sensores = 6;
long sensores[]    = {0, 0, 0, 0, 0, 0};

void setup(){
	Serial.begin(9600);
}

void loop(){
	leitura_sensores();                      // Reads sensor values and computes sensor sum and weighted average
	calculo_proporcao();                     //Calculates posicao[set point] and computes Kp,Ki and Kd
	calculo_giro();                          //Computes the error to be corrected
	atuacao_motor(velocidade_direito, velocidade_esquerdo);  //Sends PWM signals to the motors
}

void leitura_sensores() {
	media_sensores     = 0;
	somatorio_sensores = 0;
	for (int i = 0; i < quant_sensores; i++){
		sensores[i] = analogRead(i);

		media_sensores += sensores[i] * i * 1000;     //Calculating the weighted mean of the sensor readings.
		somatorio_sensores += int(sensores[i]);       //Calculating sum of sensor readings.

	}
}

void calculo_proporcao(){
	posicao   = int(media_sensores / somatorio_sensores);
	proporcao = posicao – posicao_perfeita;      // Replace posicao_perfeita by your set point
	integral  = integral + proporcao;
	derivada  = proporcao - ultima_proporcao;
	ultima_proporcao = proporcao;

	valor_erro = int(proporcao * Kp + integral * Ki + derivada * Kd);
}


void calculo_giro(){  //Restricting the error value between +256.
	if (valor_erro< -256){ valor_erro = -256;     } if (valor_erro> 256){
		valor_erro = 256;
	}

	// If valor_erro is less than zero calculate right turn speed values
	if (valor_erro< 0){
		velocidade_direito  = velocidade_maxima + valor_erro;
		velocidade_esquerdo = velocidade_maxima;
	}

	// Ifvalor_erro is greater than zero calculate left turn values

	else{
		velocidade_direito  = velocidade_maxima;
		velocidade_esquerdo = velocidade_maxima - valor_erro;
	}
}


void atuacao_motor(intvelocidade_direito, intvelocidade_esquerdo){
	// Drive motors according to the calculated values for a turn
	analogWrite(motor_direito, velocidade_direito);
	analogWrite(motor_esquerdo, velocidade_esquerdo);
	delay(60);
}

	\end{minted}
